% THIS IS SIGPROC-SP.TEX - VERSION 3.1
% WORKS WITH V3.2SP OF ACM_PROC_ARTICLE-SP.CLS
% APRIL 2009
%
% It is an example file showing how to use the 'acm_proc_article-sp.cls' V3.2SP
% LaTeX2e document class file for Conference Proceedings submissions.
% ----------------------------------------------------------------------------------------------------------------
% This .tex file (and associated .cls V3.2SP) *DOES NOT* produce:
%       1) The Permission Statement
%       2) The Conference (location) Info information
%       3) The Copyright Line with ACM data
%       4) Page numbering
% ---------------------------------------------------------------------------------------------------------------
% It is an example which *does* use the .bib file (from which the .bbl file
% is produced).
% REMEMBER HOWEVER: After having produced the .bbl file,
% and prior to final submission,
% you need to 'insert'  your .bbl file into your source .tex file so as to provide
% ONE 'self-contained' source file.
%
% Questions regarding SIGS should be sent to
% Adrienne Griscti ---> griscti@acm.org
%
% Questions/suggestions regarding the guidelines, .tex and .cls files, etc. to
% Gerald Murray ---> murray@hq.acm.org
%
% For tracking purposes - this is V3.1SP - APRIL 2009

\documentclass{acm_proc_article-sp}

%\permission{Permission to make digital or hard copies of all or part of this work for personal or classroom use is granted without fee provided that copies are not made or distributed for profit or commercial advantage and that copies bear this notice and the full citation on the first page. Copyrights for components of this work owned by others than ACM must be honored. Abstracting with credit is permitted. To copy otherwise, or republish, to post on servers or to redistribute to lists, requires prior specific permission and/or a fee. Request permissions from Permissions@acm.org.}
%\conferenceinfo{SIGSIM-PADS'15,}{June 10--12, 2015, London, United Kingdom.}
%\copyrightetc{\copyright~2015 ACM \the\acmcopyr}
%\crdata{ISBN 978-1-4503-3583-6/15/06\$15.00.\\
%DOI: http://dx.doi.org/10.1145/2769458.2769480}

% *** CITATION PACKAGES ***
\usepackage{cite}

% *** MATH PACKAGES ***
\usepackage{amsmath}
\usepackage{svg}

% *** PDF, URL AND HYPERLINK PACKAGES ***
\usepackage{url}
\usepackage{hyperref}
\usepackage{epigraph}

% *** ALGORITHM ***
\usepackage{algorithm}
\usepackage{algpseudocode}
\usepackage{hyperref}
\usepackage{booktabs}
\usepackage{enumitem}

% *** To balance reference page ***
\usepackage{flushend}

\begin{document}

\title{The {\ttlit Virtual Time Subsystem} for Linux Kernel}
\subtitle{A full implementation of this subsystem is available as VirtualTimeKernel\titlenote{https://github.com/littlepretty/VirtualTimeKernel}}
%
% You need the command \numberofauthors to handle the 'placement
% and alignment' of the authors beneath the title.
%
% For aesthetic reasons, we recommend 'three authors at a time'
% i.e. three 'name/affiliation blocks' be placed beneath the title.
%
% NOTE: You are NOT restricted in how many 'rows' of
% "name/affiliations" may appear. We just ask that you restrict
% the number of 'columns' to three.
%
% Because of the available 'opening page real-estate'
% we ask you to refrain from putting more than six authors
% (two rows with three columns) beneath the article title.
% More than six makes the first-page appear very cluttered indeed.
%
% Use the \alignauthor commands to handle the names
% and affiliations for an 'aesthetic maximum' of six authors.
% Add names, affiliations, addresses for
% the seventh etc. author(s) as the argument for the
% \additionalauthors command.
% These 'additional authors' will be output/set for you
% without further effort on your part as the last section in
% the body of your article BEFORE References or any Appendices.

\numberofauthors{2} %  in this sample file, there are a *total*
% of EIGHT authors. SIX appear on the 'first-page' (for formatting
% reasons) and the remaining two appear in the \additionalauthors section.
%
\author{
% You can go ahead and credit any number of authors here,
% e.g. one 'row of three' or two rows (consisting of one row of three
% and a second row of one, two or three).
%
% The command \alignauthor (no curly braces needed) should
% precede each author name, affiliation/snail-mail address and
% e-mail address. Additionally, tag each line of
% affiliation/address with \affaddr, and tag the
% e-mail address with \email.
%
% 1st. author
\alignauthor
Jiaqi Yan\titlenote{Also the implementer}\\
       \affaddr{Illinois Institute of Technology}\\
       \affaddr{West 31st Street, Chicago}\\
       \affaddr{Illinois, USA}\\
       \email{jyan31@hawk.iit.edu}
% 2nd. author
\alignauthor 
Dong (Kevin) Jin\\
       \affaddr{Illinois Institute of Technology}\\
       \affaddr{West 31st Street, Chicago}\\
       \affaddr{Illinois, USA}\\
       \email{dong.jin@iit.edu}
}

\date{30 July 1999}
% Just remember to make sure that the TOTAL number of authors
% is the number that will appear on the first page PLUS the
% number that will appear in the \additionalauthors section.

\maketitle

\begin{abstract}
TODO
\end{abstract}

% A category with the (minimum) three required fields
\category{I.6.3}{Simulation and Modeling}{Application}[Miscellaneous]
\category{D.4.8}{Operating Systems}{Performance}[Measurement, Simulation]
%A category including the fourth, optional field follows...
\category{D.2.8}{Software Engineering}{Metrics}[complexity measures, performance measures]

\terms{Operating System}

\keywords{Virtual Time, Linux Kernel, Emulation} % NOT required for Proceedings

\section{Introduction}
TODO

\section{The Software Architecture}

\section{Algorithms}
\algrenewcommand{\algorithmiccomment}[1]{\hskip3em$/*$ #1 $*/$}

\subsection{Algorithm for Set/Change Virtual Time Dilation}
\begin{algorithm*}[t]
\caption{Set Time Dilation Factor}
\label{Alg-SetTDF}
\begin{algorithmic}[1]
\Function{set\_dilation}{$tsk, \;new\_tdf$}\Comment{Set $new\_tdf$ to process $tsk$}
\State $old\_tdf \gets tsk.dilation$
\State $vsn \gets tsk.virtual\_start\_nsec$
\If{$new\_tdf = old\_tdf$}
	\State return 0\Comment{do nothing}
\ElsIf{$old\_tdf = 0$}
	\State return \texttt{\uppercase{init\_virtual\_time}($tsk, new\_tdf$)}\Comment{enter virtual time with $new\_tdf$}
\ElsIf{$new\_tdf = 0$}
	\State return \texttt{\uppercase{clean\_up\_virtual\_time}($tsk$)}\Comment{exit virtual time}
\ElsIf{$new\_tdf > 0$}\Comment{real works here}
	\State $tsk.dilation \gets 0$
	\State $tsk.virtual\_start\_nsec \gets 0$
	\State \texttt{\_\_getnstimeofday}(\&$ts$)\Comment{$ts$: a \texttt{timespec} temp}
	\State $now \gets$ \texttt{timespec\_to\_ns}(\&$ts$)
	\State $tsk.virtual\_start\_nsec \gets vsn$
	\State $delta\_ppn \gets now \; - \; tsk.physical\_past\_nsec \; - \; tsk. physical\_start\_nsec \; - \; tsk. freeze\_past\_nsec$
	\State $delta\_vpn \gets delta\_ppn \; / \; old\_tdf$
	\State $tsk.virtual\_past\_nsec \; += \; delta\_vpn$
	\State $tsk.physical\_start\_nsec \gets (now \; - \; tsk\rightarrow freeze\_past\_nsec)$
	\State $tsk.physical\_past\_nsec \gets 0$
	\State $tk.dilation \gets new\_tdf$
	\State return 0
\Else
	\State return \texttt{-EINVAL}
\EndIf
\EndFunction
\end{algorithmic}
\end{algorithm*}

\subsection{Algorithm for Freeze and Unfreeze}
\begin{algorithm*}[t]
\caption{Freeze and Unfreeze Process}%: Set \texttt{new\_tdf} to process \textt{tsk}
\label{Alg-Freeze}
\begin{algorithmic}[1]
\Function{Freeze}{tsk}
\State \texttt{kill\_pgrp(task\_pgrp($tsk$), SIGSTOP, 1)}
\State \texttt{\_\_getnstimeofday}(\&$ts$)\Comment{$ts$: a \texttt{timespec} temp}
\State $now \gets$ \texttt{timespec\_to\_ns}(\&$ts$)
\State $tsk.freeze\_start\_nsec \gets now$
\EndFunction
\\
\Function{Unfreeze}{tsk}
\State \texttt{\_\_getnstimeofday}(\&$ts$)\Comment{$ts$: a \texttt{timespec} temp}
\State $now \gets$ \texttt{timespec\_to\_ns}(\&$ts$)
\State $tsk.freeze\_past\_nsec \gets now \; - \; tsk.freeze\_start\_nsec$
\State $tsk.freeze\_start\_nsec \gets 0$
\State \texttt{kill\_pgrp(task\_pgrp($tsk$), SIGCONT, 1)}
\EndFunction
\end{algorithmic}
\end{algorithm*}

\subsection{Algorithm for Virtual Time Keeping}
\begin{algorithm*}[t]
\caption{Time Dilation Algorithm}%: Dilate \texttt{ts} if a current process uses virtual clock}
\label{Alg-DilateTimeKeeping}
\begin{algorithmic}[1]
\Function{init\_virtual\_time}{$tsk,\;tdf$}
\If{$tdf>0$}
    \State $tk.dilation \gets tdf$
    \State $tk.virtual\_start\_nsec \gets 0$
    \State \texttt{\_\_getnstimeofday}$(\&ts)$\Comment{$ts$: a \texttt{timespec} temp}
    \State $tk.virtual\_start\_nsec \gets $\texttt{timespec\_to\_ns}$(\&ts)$
    \State $tk.physical\_past\_nsec \gets 0$
    \State $tk.virtual\_past\_nsec \gets 0$
\EndIf
\EndFunction
\\
\Function{update\_physical\_time}{$tsk, \;ts$}
\EndFunction
\\
\Function{update\_virtual\_time}{$delta\_ppn, \;tdf$}
\EndFunction
\\
\Function{do\_dilatetimeofday}{\texttt{struct timespec *ts}}
\State{Let \texttt{p} denote the current process using virtual time} %system-wide}
\If{\texttt{p$\rightarrow$virtual\_start\_nsec > 0  and p$\rightarrow$dilation > 0}}
	\State \texttt{now = timespec\_to\_ns(ts)}
	\State \texttt{physical\_past\_nsec = now - p$\rightarrow$virtual\_start\_nsec}
	\State \texttt{virtual\_past\_nsec = (physical\_past\_nsec - p$\rightarrow$physical\_past\_nsec) / p$\rightarrow$dilation + p$\rightarrow$virtual\_past\_nsec}\Comment{virtual time computation}
	\State \texttt{dilated\_now = virtual\_past\_nsec + p$\rightarrow$virtual\_start\_nsec}
	\State \texttt{dilated\_ts = ns\_to\_timespec(dilated\_now)}
	\State \texttt{ts$\rightarrow$tv\_sec = dilated\_ts.tv\_sec}
	\State \texttt{ts$\rightarrow$tv\_nsec = dilated\_ts.tv\_nsec}
	\State \texttt{p$\rightarrow$physical\_past\_nsec = physical\_past\_nsec}\Comment{update process's physical time}
	\State \texttt{p$\rightarrow$virtual\_past\_nsec = virtual\_past\_nsec}\Comment{update process's virtual time}
\EndIf
\EndFunction
\end{algorithmic}
\end{algorithm*}

\section{Tests and Overhead Measurements}

\section{Conclusions}
This paragraph will end the body of this sample document.
Remember that you might still have Acknowledgments or
Appendices; brief samples of these
follow.  There is still the Bibliography to deal with; and
we will make a disclaimer about that here: with the exception
of the reference to the \LaTeX\ book, the citations in
this paper are to articles which have nothing to
do with the present subject and are used as
examples only\cite{Lamport:LaTeX}.
%\end{document}  % This is where a 'short' article might terminate

%ACKNOWLEDGMENTS are optional
\section{Acknowledgments}
This section is optional; it is a location for you
to acknowledge grants, funding, editing assistance and
what have you.  In the present case, for example, the
authors would like to thank Gerald Murray of ACM for
his help in codifying this \textit{Author's Guide}
and the \textbf{.cls} and \textbf{.tex} files that it describes.

%
% The following two commands are all you need in the
% initial runs of your .tex file to
% produce the bibliography for the citations in your paper.
\bibliographystyle{abbrv}
\bibliography{sigproc}  % sigproc.bib is the name of the Bibliography in this case
% You must have a proper ".bib" file
%  and remember to run:
% latex bibtex latex latex
% to resolve all references
%
% ACM needs 'a single self-contained file'!
%
%APPENDICES are optional
%\balancecolumns
\appendix
%Appendix A
\section{Headings in Appendices}
The rules about hierarchical headings discussed above for
the body of the article are different in the appendices.
In the \textbf{appendix} environment, the command
\textbf{section} is used to
indicate the start of each Appendix, with alphabetic order
designation (i.e. the first is A, the second B, etc.) and
a title (if you include one).  So, if you need
hierarchical structure
\textit{within} an Appendix, start with \textbf{subsection} as the
highest level. Here is an outline of the body of this
document in Appendix-appropriate form:
\subsection{Introduction}
\subsection{The Body of the Paper}
\subsubsection{Type Changes and  Special Characters}
\subsubsection{Math Equations}
\paragraph{Inline (In-text) Equations}
\paragraph{Display Equations}
\subsubsection{Citations}
\subsubsection{Tables}
\subsubsection{Figures}
\subsubsection{Theorem-like Constructs}
\subsubsection*{A Caveat for the \TeX\ Expert}
\subsection{Conclusions}
\subsection{Acknowledgments}
\subsection{Additional Authors}
This section is inserted by \LaTeX; you do not insert it.
You just add the names and information in the
\texttt{{\char'134}additionalauthors} command at the start
of the document.
\subsection{References}
Generated by bibtex from your ~.bib file.  Run latex,
then bibtex, then latex twice (to resolve references)
to create the ~.bbl file.  Insert that ~.bbl file into
the .tex source file and comment out
the command \texttt{{\char'134}thebibliography}.
% This next section command marks the start of
% Appendix B, and does not continue the present hierarchy
\section{More Help for the Hardy}
The acm\_proc\_article-sp document class file itself is chock-full of succinct
and helpful comments.  If you consider yourself a moderately
experienced to expert user of \LaTeX, you may find reading
it useful but please remember not to change it.
\balancecolumns
% That's all folks!
\end{document}
